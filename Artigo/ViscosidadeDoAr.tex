\documentclass[a4paper, 12pt]{article}
\usepackage[top=1cm, bottom=1.5cm, left=1cm, right=1cm]{geometry}
\usepackage[utf8]{inputenc}
\usepackage{graphicx, caption}
\usepackage{float}
\usepackage{amsmath, amsfonts, amssymb, esint}
\usepackage{multicol}
\usepackage{titlesec}
\usepackage{hyperref}
\usepackage{indentfirst}

\pagenumbering{Roman}

\titleformat{\section}{\normalfont\bfseries}{\thesection.}{0.5em}{}

\hypersetup
{
    colorlinks=true,
    linkcolor=blue,
    filecolor=magenta,      
    urlcolor=blue,
}

\begin{document}
	\begin{center}
	\begin{large}
		\textbf{Determinação da Viscosidade do Ar}	
	\end{large}
	\end{center}
	
	\begin{center}
		\textit{Iago Braz Mendes, Hugo Danilo Santos Alkimim, Gabriel Oliveira Mota, e Marcos Aurélio Duarte Carvalho}
	\end{center}

	\section*{Resumo}
		Neste artigo, explora-se um arranjo experimental de um oscilador massa-mola para investigar a oscilação harmônica amortecida. A determinação da amplitude da oscilação como uma função temporal foi determinada precisa e economicamente, utilizando vídeos obtidos por meio de um celular, a ferramenta \textit{Tracker -- Video Analysis and Modeling Tool for Physics Education --} e algoritmos em \textit{MATLAB}. Assim, foi possível determinar o coeficiente de amortecimento, que foi usado para encontrar a viscosidade do ar.

	\begin{multicols}{2}
		\setlength{\parindent}{4ex}
		\section{Introdução}
			\par A teoria acerca de sistemas oscilatórios amortecidos é aprofundada por alguns artigos [\textit{referência(s)}]. Contudo, é comum encontrar discrepâncias entre as previsões e os resultados obtidos em laboratórios. Nesse sentido, este artigo apresenta um modelo experimental de fácil implementação para determinar a viscosidade do ar a partir de uma oscilação massa-mola amortecida. Com os dados coletados, fazemos 3 análises a fim de encontrar qual método mais se aproxima da realidade.
			\par Na seção \hyperref[sec:teoria]{\ref{sec:teoria}}, a parte teórica sobre viscosidade e oscilações amortecidas é rapidamente exposta. Na seção \hyperref[sec:experimento]{\ref{sec:experimento}}, o modelo experimental usado para obter dados é explicado. Na seção \hyperref[sec:analise]{\ref{sec:analise}}, a análise de dados e seus resultados são discutidos, abrangendo 3 métodos distintos. Por fim, as conclusões são apresentadas na seção \hyperref[sec:conclusao]{\ref{sec:conclusao}}.
			
		\section{Teoria} \label{sec:teoria}
			\par Em primeira instância, viscosidade ($\eta$) é a propriedade física que descreve a resistência de um fluido para escoar, que é geralmente expressa usando a Lei de Newton da Viscosidade:
			\begin{figure}[H]
				\centering
				\includegraphics[scale=0.6]{./img/viscosidade.png}
			\end{figure}
			\begin{equation}
				\frac{F}{A} = \eta \dfrac{dv_x}{dy}
			\end{equation}
			em que $F$ é a força aplicada, $A$ é a área em $y$ e $z$, e $\dfrac{dv_x}{dy}$ é a derivada espacial da velocidade. No Sistema Internacional, a viscosidade possui \textit{pascal-segundo} [$Pa \cdot s$] como unidade de medida.
			\par Além disso, quando estudamos a mecânica de fluidos, devemos sempre considerar o número de Reynolds, um coeficiente adimensional utilizado para o cálculo do regime de escoamento de um fluido sobre uma superfície. Esse coeficiente é dado por:
			\begin{equation}
				Re = \frac{\rho v l}{\eta}
			\end{equation}
			em que $\rho$ é a densidade do fluido e $l$ é a dimensão linear característica do objeto oscilante (para uma esfera, $l = 2 r$, em que $r$ é o raio).
			\par Finalmente, quando analizamos oscilações harmônicas simples no mundo real, precisamos considerar a energia dissipada devido à força de atrito. Para pequenos números de Reynolds, a força de atrito na esfera é dada pela Lei de Stoke:
			\begin{equation}
				F = 6 \pi r \eta v
			\end{equation}
			\par Nesse caso, a equação do movimento para uma esfera com massa $m$ e uma mola com constante elástica $k$ é dada por
			\begin{equation}
				\ddot{x} + 2 \gamma \dot{x} + \omega_0^2 x = 0
			\end{equation}
			em que as constantes são dadas por
			\begin{equation} \label{eq:constantes-stokes}
				\omega_0^2 = \frac{k}{m} \qquad \gamma = \frac{3 \pi \eta r}{m}
			\end{equation}
			\par Portanto, quando $\omega_0 > \gamma$ (caso de sub-amortecimento), a solução é
			\begin{equation}
				x = A \, e^{- \gamma t} \, \cos (\omega t + \phi)
			\end{equation}
			em que $A$, $\phi$ e $\omega$ são, respectivamente, a amplitude, a fase, e a frequência da oscilação.
			\par Contudo, quando os números de Reynolds não são pequenos, podemos usar a equação desenvolvida por Landau e Lifshitz:
			\small \begin{equation} \label{eq:landau}
				F = 6 \pi \eta r \left(1 + \frac{r}{\delta} \right) v + 3 \pi r^2 \left( 1 + \frac{2 r}{9 \delta} \right) \rho \delta \dfrac{d v}{d t}
			\end{equation} \normalsize
			em que $\delta$ a profundidade de penetração dentro do fluido ao redor do objeto oscilante, que é dada por
			\begin{equation}
				\delta = \sqrt{\frac{2 \eta}{\rho \omega}}
			\end{equation}
			\par Quando resolvemos essa equação, encontramos uma solução análoga à anterior:
			\begin{equation} \label{eq:posicao}
				x = A \, e^{- \gamma t} \, \cos (\omega t + \phi)
			\end{equation}
			\par Todavia, agora as constantes são determinadas por
			\begin{equation} \label{eq:constantes-landau} \begin{split}
				\omega_0^2 = \frac{k}{f_2 3 \pi r^2 \left(1 + \frac{2 r}{9 \delta} \right) + m} \\
				\gamma = \frac{3 \pi \eta r \left(1+ \frac{r}{\delta} \right)}{f_1 \left[ f_2 3 \pi r^2 \left(1+\frac{2 r}{9 \delta} \right) \rho \delta + m \right]}
			\end{split} \end{equation}
			em que $f_1$ e $f_2$ são coeficientes semi-empíricos.
			\par Neste artigo, fazemos uso somente dos pontos máximos da oscilação (ou seja, das amplitudes). Com isso, a equação \hyperref[eq:posicao]{\ref{eq:posicao}} pode ser reescrita da seguinte forma:
			\begin{equation} \label{eq:amplitudes}
				A = A_0 e^{- \gamma t} \quad \therefore \quad \ln A = - \gamma t + \ln A_0
			\end{equation}
			em que $A$ é a amplitude em função do tempo e $A_0$ é a amplitude inicial.
			
		\section{Experimento} \label{sec:experimento}
			\par A configuração experimental é simples e pode ser facilmente repetida. Ela consiste de uma bola com uma massa conhecida, uma mola, um suporte com uma escala de comprimento, um dispositivo para gravar a oscilação, um suporte para o dispositivo, e um cronômetro.
			\par Inicialmente, fizemos o experimento usando uma bola de tênis (\hyperref[img:tenis]{Figura 1}) e filmamos toda a oscilação. Após analisar os dados dessa configuração, optamos por utilizar um objeto de estudo com menor rugosidade superficial, visando a adequar nosso experimento aos modelos matemáticos utilizados. Além disso, optamos por gravar trechos mais curtos, a fim de obter vídeos que representem a oscilação em cada minuto, facilitando a análise de dados.
			\par Portanto, fizemos um segundo experimento usando uma bola de metal (\hyperref[img:metal]{Figura 2}) e filmamos entre 5 e 10 segundos em intervalos de 1 minuto.
			\par Por fim, uma terceira configuração foi estabelecida, em que o movimento amortecido de uma bola de metal (similar à \hyperref[img:metal]{Figura 2}) foi gravado num período de 30 minutos sem interrupções.
			\begin{figure}[H] \label{img:tenis}
				\centering
				\includegraphics[scale=0.09]{./img/bolaTenis.jpg}
				\captionsetup{labelformat=empty}
				\caption{\textbf{Figura 1:} Experimento com bola de tênis.}
			\end{figure}
			\begin{figure}[H] \label{img:metal}
				\centering
				\includegraphics[scale=0.3]{./img/bolaMetal.png}
				\captionsetup{labelformat=empty}
				\caption{\textbf{Figura 2:} Experimento com bola de metal.}
			\end{figure}
			
		\section{Análise de Dados} \label{sec:analise}
			\par Com os dados da segunda configuração, fizemos duas análises para determinar o valor da viscosidade do ar: utilizando a maior amplitude de cada vídeo e fazendo a média aritmética das amplitudes atingidas em cada trecho.
			\par No primeiro método, analisamos cada gravação -- quadro a quadro --, com o auxílio de um editor de vídeo (\hyperref[img:quadros]{Figura 3}), para encontrar a maior amplitude atingida e o correspondente tempo no cronômetro em tela. Os dados coletados podem ser vistos na \hyperref[img:m1]{Figura 4}. Usando a relação \hyperref[eq:amplitudes]{\ref{eq:amplitudes}}, temos que a constante $\gamma \simeq 2.761 \cdot 10^{-3}$. Com isso, a viscosidade foi calculada usando as constantes mostradas na equação \hyperref[eq:constantes-stokes]{\ref{eq:constantes-stokes}} e na equação \hyperref[eq:constantes-landau]{\ref{eq:constantes-landau}}, tendo como resultados -- respectivamente -- $\#.\#\#\# \cdot 10^{-\#} \, Pa \cdot s$ e $\#.\#\#\# \cdot 10^{-\#} \, Pa \cdot s$.
			\par No segundo método, usamos o programa \textit{Tracker -- video analysis and modelling tool --} para conseguir mais posições em cada vídeo e depois calcular a média aritmética das amplitudes atingidas nesse intervalo. Os dados coletados podem ser vistos na \hyperref[img:m2]{Figura 5}. Usando a relação \hyperref[eq:amplitudes]{\ref{eq:amplitudes}}, temos que a constante $\gamma \simeq 2.313 \cdot 10^{-3}$. Com isso, a viscosidade foi calculada usando as constantes mostradas na equação \hyperref[eq:constantes-stokes]{\ref{eq:constantes-stokes}} e na equação \hyperref[eq:constantes-landau]{\ref{eq:constantes-landau}}, tendo como resultados -- respectivamente -- $\#.\#\#\# \cdot 10^{-\#} \, Pa \cdot s$ e $\#.\#\#\# \cdot 10^{-\#} \, Pa \cdot s$.
			\par Além disso, com os dados da terceira configuração, um algoritmo escrito em MatLab (disponível \href{https://github.com/hugoalkimim/ViscosidadeDoAr/tree/master/Algoritmo}{aqui}) foi utilizado para encontrar os máximos e, consequentemente, encontrar a constante $\gamma \simeq 2.367 \cdot 10^{-4}$ (usando a relação \hyperref[eq:amplitudes]{\ref{eq:amplitudes}}). Os dados coletados podem ser vistos na \hyperref[img:m3]{Figura 6}. Com isso, a viscosidade foi calculada usando as constantes mostradas na equação \hyperref[eq:constantes-stokes]{\ref{eq:constantes-stokes}} e na equação \hyperref[eq:constantes-landau]{\ref{eq:constantes-landau}}, tendo como resultados -- respectivamente -- $\#.\#\#\# \cdot 10^{-\#} \, Pa \cdot s$ e $\#.\#\#\# \cdot 10^{-\#} \, Pa \cdot s$.
			\begin{figure}[H] \label{img:quadros}
				\centering
				\includegraphics[scale=0.4]{./img/quadros.png}
				\captionsetup{labelformat=empty}
				\caption{\textbf{Figura 3:} Análise da gravação quadro a quadro.}
			\end{figure}
			\begin{figure}[H] \label{img:m1}
				\centering
				\includegraphics[scale=0.3]{./img/m1.png}
				\captionsetup{labelformat=empty}
				\caption{\textbf{Figura 4:} Gráfico com dados do primeiro método.}
			\end{figure}
			\begin{figure}[H] \label{img:m2}
				\centering
				\includegraphics[scale=0.3]{./img/m2.png}
				\captionsetup{labelformat=empty}
				\caption{\textbf{Figura 5:} Gráfico com dados do segundo método.}
			\end{figure}
			\begin{figure}[H] \label{img:m3}
				\centering
				\includegraphics[scale=0.3]{./img/m3.png}
				\captionsetup{labelformat=empty}
				\caption{\textbf{Figura 6:} Gráfico com dados do terceiro método.}
			\end{figure}
			
		\section{Conclusão} \label{sec:conclusao}
		
		\section{Referências}
		
	\end{multicols}
\end{document}
