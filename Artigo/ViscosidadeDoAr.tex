\documentclass[a4paper, 12pt]{article}
\usepackage[top=1cm, bottom=1.5cm, left=1cm, right=1cm]{geometry}
\usepackage[utf8]{inputenc}
\usepackage{graphicx}
\usepackage{float}
\usepackage{amsmath, amsfonts, amssymb, esint}
\usepackage{multicol}
\pagenumbering{Roman}
\begin{document}
	\begin{center}
	\begin{large}
		\textbf{Determinação da Viscosidade do Ar}	
	\end{large}
	\end{center}
	
	\begin{center}
		\textit{Gabriel Oliveira Mota, Iago Braz Mendes, e Marcos Aurélio Duarte Carvalho a}
	\end{center}

	\textbf{Resumo}
		
		
		Neste artigo, explora-se um arranjo experimental de um oscilador massa-mola para investigar a oscilação harmônica amortecida. A determinação da amplitude da oscilação como uma função temporal foi determinada precisa e economicamente, utilizando a ferramente \textit{Tracker -- Video Analysis and Modeling Tool for Physics Education --} e vídeos obtidos por meio de um celular. Assim, foi possível determinar o coeficiente de amortecimento, que foi usado para determinar a viscosidade do ar. Sob a supervisão do professor mestre Marcos Aurélio Duarte Carvalho, esse experimento tem sido usado em disciplinas como Laboratório de Mecânica para estudantes de engenharia.
	
	\begin{multicols}{2}
	\begin{enumerate}
		\item \textbf{Introdução}
			
		\item \textbf{Teoria}
			
			
			Em primeira instância, viscosidade ($\eta$) é basicamente a quantidade que descreve a resistência de um fluido para escoar, que é geralmente expressa usando a Lei de Newton da Viscosidade:
				$$\frac{F}{A} = \eta \dfrac{dv_x}{dy}$$
				em que $F$ é a força aplicada, $A$ é a área em $y$ e $z$, e $\dfrac{dv_x}{dy}$ é a derivada espacial da velocidade. No Sistema Internacional, a viscosidade possui \textit{pascal-segundo} [$Pa \cdot s$] como unidade de medida. \linebreak
				\linebreak Além disso, quando estudamos a mecânica de fluidos, devemos sempre considerar os números de Reynolds, que são dados por
				$$Re = \frac{\rho v l}{\eta}$$
				em que $\rho$ é a densidade do fluido e $l$ é a dimensão linear característica do objeto oscilante (para uma esfera, $l = 2 r$, em que $r$ é o raio). \linebreak
				\linebreak Finalmente, quando analizamos oscilações harmônicas simples no mundo real, precisamos considerar a energia dissipada devido à força de atrito. Para pequenos números de Reynolds, a força de atrito na esfera é dada pela Lei de Stoke:
				$$F = 6 \pi r \eta v$$
				Neste caso, a equação do movimento para uma esfera com massa $m$ e uma mola com constante elástica $k$ é dada por
				$$\ddot{x} + 2 \gamma \dot{x} + \omega_0^2 x = 0$$
				em que as constantes são dadas por
					$$\omega_0^2 = \frac{k}{m} \qquad \gamma = \frac{3 \pi \eta r}{m}$$
				Portanto, quando $\omega_0 > \gamma$ (\textit{under-damped case}), a solução é
					$$x = A \, e^{- \gamma t} \, \cos (\omega t + \phi)$$
				em que $A$, $\phi$ e $\omega$ são a amplitude, a fase, e a frequência, respectivamente, da oscilação. \linebreak
				\linebreak Contudo, quando os números de Reynolds não são pequenos, podemos usar a equação desenvolvida por Landau e Lifshitz:
				$$F = 6 \pi \eta r \left(1 + \frac{r}{\delta} \right) v + 3 \pi r^2 \left( 1 + \frac{2 r}{9 \delta} \right) \rho \delta \dfrac{d v}{d t}$$
				em que $\delta$ a profundidade de penetração dentro do fluido ao redor do objeto oscilante, que é dada por
				$$\delta = \sqrt{\frac{2 \eta}{\rho \omega}}$$
				Quando resolvemos essa equação, encontramos uma solução análoga à anterior:
					$$x = A \, e^{- \gamma t} \, \cos (\omega t + \phi)$$
				Todavia, agora as constantes são determinadas por
					$$\gamma = \frac{3 \pi \eta r \left(1+ \frac{r}{\delta} \right)}{f_1 \left[ f_2 3 \pi r^2 \left(1+\frac{2 r}{9 \delta} \right) \rho \delta + m \right]}$$
					$$\omega_0^2 = \frac{k}{f_2 3 \pi r^2 \left(1 + \frac{2 r}{9 \delta} \right) + m}$$
				em que $f_1$ e $f_2$ são coeficientes semi-empíricos. \linebreak
		\item \textbf{Experimento}
			
			
			A configuração experimental é simples e pode ser facilmente repetida (figures 1 e 2). Ela consiste de uma bola com uma massa conhecida, uma mola, um suporte com uma escala de comprimento, um dispositivo para gravar partes da oscilação, um suporte para o dispositivo, e um cronômetro que nos dá o tempo real no vídeo em câmera lenta gerado. \linebreak
			\linebreak Inicialmente, nós fizemos o experimento usando uma bola de tênis e filmando toda a oscilação (figure 1). Depois de fazermos a análise de dados dessa configuração, decidimos que seria melhor usar uma bola mais lisa e gravar mais separadamente. \linebreak
			Portanto, nós fizemos um segundo experimento usando uma bola de metal (figure 2) e filmando entre 5 e 10 segundos em intervalos de 1 minuto.\linebreak
			
			\begin{figure}[H]
			\centering
			\includegraphics[scale=0.09]{1A.jpg}
			\caption{Experimento com bola de tênis}
			\end{figure}
			
			\begin{figure}[H]
			\centering
			\includegraphics[scale=0.3]{1B.png}
			\caption{Experimento com bola de metal}
			\end{figure}
			
		\item \textbf{Análise de Dados}
		
		
			Com os dados da última configuração, fizemos 2 análises para conseguir o valor mais próximo da real viscosidade do ar: pegando a maior amplitude de cada vídeo e fazendo a média aritmética das amplitudes em cada vídeo. \linebreak
			Na primeira análise, tivemos que olhar cada gravação -- frame por frame -- para encontrar a maior amplitude e o tempo no cronômetro em que ela foi atingida. Com isso, conseguimos encontrar o valor da viscosidade do ar como sendo $2.78 \cdot 10^{-5} \, Pa \cdot s$ (ignorando as incertezas). Apesar do fato de que esse processo foi efetivo, ele poderia ser mais preciso. Assim, decidimos encontrar amplitudes mais exatas. \linebreak
			\linebreak Na segunda análise, usamos um programa chamado \textit{Tracker -- video analysis and modelling tool --} para conseguir mais posições em cada vídeo e depois calcular a média aritmética das amplitudes atingidas nesse intervalo. Como consequência dessa nova análise, conseguimos determinar o valor da viscosidade do ar como $2.02 \cdot 10^{-5} \, Pa \cdot s$ (ignorando a incerteza novamente), o qual está bem próximo do real ($1.81 \cdot 10^{-5} \, Pa \cdot s$). \linebreak
		\item \textbf{Conclusão}
		\item \textbf{Referências}
	\end{enumerate}
	\end{multicols}
\end{document}